\setbeamertemplate{background}[bgfirst]
\setbeamertemplate{footline}[first]
\subtitle{Bioinformatik}
\titlegraphic{Kapitel/Einfuehrung/Bilder/saelogo.jpg}
\begin{frame}[noframenumbering]
    \titlepage
    \begin{textblock}{10}(4.75,15)
    \cite{saelogo}
    \end{textblock}
\end{frame}
\setbeamertemplate{footline}[presentationbody] 
\setbeamertemplate{background}[bgbody]

\begin{frame}{Vorstellung}
    \note<4->{
        Erstes Mal Vorlesung an der HTW\\Bitte um Geduld, Hinweise\\
        \begin{itemize}
            \item Lerne erst Modalitäten kennen
            \item Kenne die Vorkenntnisse/präferierte Art der Studierenden nicht
            \item Primäres Ziel: Verständnis aufbauen, "durch den Stoff kommen" zweitrangig
            \item Also bitte (rechtzeitig) melden, wenn Interesse an Vertiefung besteht
            \item Gemeinsame Reise - bitte immer melden, wenn was ist!
            \item Gerne auch außerhalb der Sprechzeiten, wenn die Tür offen ist einfach reinkommen.
        \end{itemize}
        Whiteboard:
        \begin{itemize}
            \item Wer nicht aufgerufen werden will, nach der VL bei mir melden (stelle keine Fragen) oder X oben links bei Antwort
            \item Spontanes, prägnantes, kohärentes Sprechen wichtig! Beispiel: Empfehlung an Politikerin
        \end{itemize}
    }
    \begin{itemize}
        \item<2-> Kurz zu mir
        \only<2-5>{
            \begin{itemize}
                \item<2-5> Kontakt: Piotr.Dabrowski@htw-berlin.de, Sprechstunde Montag 10-11 - gerne nutzen!
                \item<2-5> Github: https://github.com/dabrowskiw/
                \item<3-5> Geboren 1981 in Warschau
                \item<3-5> Studium der Biotechnologie \& Informatik an der TU Berlin
                \item<3-5> Promotion über Auswertung von Hochdurchsatzdaten für Virus-Diagnostik
                \item<3-5> Aufbau der bioinformatischen Analytik für das NGS-Labor des RKI
                \item<3-5> Aufbau der Bioinformatics Core Facility am RKI
                \item<4-5> Seit WS 2019/2020 an der HTW
                \item<5> Hang zum Experimentieren in der Vorlesung - Feedback erwünscht!
            \end{itemize}
        }
        \item<6-> Der Todesstern \& Mini-Whiteboards
    \end{itemize}
\end{frame}

\begin{frame}{Benotung}
	TODO
	\begin{itemize}
		\item Projektaufgabe (Abgabe 29.01.2020 24:00)
		\item Klausur
	\end{itemize}
\end{frame}
