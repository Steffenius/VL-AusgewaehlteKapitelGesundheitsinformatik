\subsection{First Generation Sequencing}

Mit der PCR kann nur der Nachweis erbracht werden, ob eine von zwei konservierten Regionen (zu denen die revers-komplementären Primer synthetisiert werden konnten) flankierte DNA-Region in einer Probe existiert. Um die Sequenz dieser DNA-Region zu entschlüsseln wird Sequenzierung eingesetzt. 

\subsubsection{Sanger-Sequenzierung}

Die älteste bekannte Sequenziermethode ist die von Frederick Sanger im Jahre 1977 publizierte \textbf{Sanger-Sequenzierung}. Diese läuft ähnlich ab, wie die PCR: Es werden zu einer vorab mittels PCR amplifizierten Probe Polymerase, Nukleotide sowie einer der bei der PCR verwendeten Primer Primer (direkt neben der zu sequenzierenden DNA-Region liegend - anders als bei der PCR, wo zwei links und rechts flankierende Primer eingesetzt werden) hinzugegeben. Dazu kommt noch eine geringe Menge modifizierter Nukleotide: Diese sind fluoreszent markiert (jede Base trägt eine Farbe, es werden also vier unterschiedliche Farben verwendet) und zusätzlich chemisch so modifiziert, dass die Polymerase keine weiteren Nukleotide an ein modifiziertes Nukleotid anbringen kann. Beginnt man nun den Vorgang der PCR, kann sich der Primer an die vorab amplifizierte Region anlagern und die Polymerase beginnt die dahinter liegende Sequenz zu kopieren. Verwendet sie dabei aber an irgendeiner zufälligen Stelle ein modifiziertes Nukleotid, muss sie den Kopiervorgang abbrechen. Es entsteht ein DNA-Fragment, das mit einer modifizierten Base irgendwo im Bereich der zu analysierenden Sequenz aufhört und in einer dieser Base entsprechenden Farbe fluoreszent markiert ist. Da sehr viele Kopien der zu analysierenden DNA-Sequenz vorhanden sind (Ausgangsmaterial ist ja eine vorab per PCR amplifizierte Probe) und der Einbau modifizierter Basen an einer zufälligen Position stattfindet, liegt am Ende mit an Sicherheit grenzender Wahrscheilichkeit für jede Base der zu analysierenden Sequenz ein DNA-Fragment vor, das mit genau dieser Base endet und entsprechend markiert ist. Werden diese Fragmente nun in einer Kapillare der Länge nach sortiert\footnote{Das geschieht mittels Elektrophorese.}, kann über die Reihenfolge, in der die einzelnen fluoreszenten Markierungen in der Kapillare vorliegen die ursprüngliche Sequenz rekonstruiert werden. Mit dieser Methode lassen sich Sequenzen von bis zu ca. 1000 Basen Länge sequenzieren. 

\subsubsection{Pyrosequenzierung}

Eine modernere Sequenziermethode ist die \textbf{Pyrosequenzierung}. Ähnlich wie bei der Sanger-Sequenzierung ist das Ausgangsmaterial ein PCR-Produkt, und die Sequenzierreaktion selber ähnelt der PCR-Reaktion. Im Gegensatz zur Sanger-Sequenzierung wird allerdings hier die Geschwindigkeit der Synthese des Gegenstranges streng kontrolliert: Zunächst werden zu dem zu sequenzierenden PCR-Produkt nur Primer, Polymerase sowie ein paar weitere, für die spätere Signalgenerierung notwendige Enzyme hinzugegeben. Dann wird nur ein Typ Nukleitid hinzugefügt - also nur A, T, G oder C, anstatt wie bei der PCR oder Sanger-Sequenzierung ein Mix aller Nukleotide. Kann dieses Nukleotid von der Polymerase eingebaut werden, werden die bei dem Einbau entstehenden Nebenprodukte von den zusätzlichen Enzymen verwendet, um ein Lichtsignal zu generieren\footnote{Die hier verwendeten Enzyme sind die gleichen, die von Glühwürmchen verwendet werden, um Licht zu generieren. Man macht sich zu Nutze, dass das bei der Verknüpfung zweier Basen anfallende Nebenprodukt das gleiche ist, welches die Glühwürmchen auch als Ausgangspunkt in ihrer Lichterzeugungs-Reaktion verwenden. Dieser Ausgangsstoff ist das Pyrophosphat, daher der Name der Sequenziermethode.}. Dieses Lichtsignal kann mit einer Kamera aufgenommen werden. Danach werden die überschüssigen Nukleotide entfernt und ein anderes Nukleotid wird hinzugefügt. Daraus, bei welcher Nukleotidzugabe Lichtsignale entstehen, kann die Sequenz rekonstruiert werden. Die Pyrosequenzierung erlaubt es zwar, im Gegensatz zur Sanger-Sequenzierung, den Sequenziervorgang in echtzeit mitzuverfolgen, allerdings können mit dieser Methode nur ca. 200 Basen lange Sequenzen generiert werden, bevor das Signal zu Rauschen degeneriert. 
