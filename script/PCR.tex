\subsection{PCR}

Die Fähigkeit von Organismen, ihre DNA zu kopieren, wird in der molekularen Analytik dafür genutzt, DNA zu manipulieren und nachzuweisen. Eine der grundlegenden Methoden dabei ist die \textbf{Polymerase Chain Reaction} (\textbf{PCR}). Die PCR wird verwendet, um gezielt ein bekanntes Stück DNA massiv zu vermehren. Da es möglich ist, die Menge an DNA in einer Probe zu messen, kann der Erfolg dieser Vermehrung gemessen werden. Wird also versucht, in einer Probe gezielt ein Stück DNA zu vermehren, von dem bekannt ist, dass es nur in einem bestimmten Organismus vorhanden ist (z.B. in Ebolaviren), und wird danach gezeigt, dass die Vermehrung erfolgreich war, ist der Beweis erbracht, dass dieser Organismus in der Probe vorhanden ist. 

Die PCR bedient sich der DNA-Polymerase und ihrer zwei definierenden Eigenschaften: Der Fähigkeit, einen DNA-Einzelstrang zu kopieren, und ihrer Abhängigkeit von einem Primer. Um eine PCR durchzuführen, werden zu einer Probe mit unbekannter DNA zunächst DNA-Polymerase sowie Nukleotide\footnote{inklusive einiger weiterer für die Reaktion notwendiger Chemikalien} hinzugefügt. Da keine Primase vorhanden ist, kann die DNA-Polymerase in diesem Zustand noch keine DNA duplizieren. Anstatt von Primase werden nun aber direkt Primer hinzugefügt: Kurze, meist ca. 20 Basen lange DNA-Fragmente, die künstlich synthetisiert wurden und revers-komplementär zu nah beieinander liegenden DNA-Bereichen sind, welche nur in dem nachzuweisenden Organismus vorhanden sind. Wird nun die Temperatur soweit erhöht, dass die DNA-Doppelhelix sich in zwei Einzelstränge löst, und danach wieder ein wenig reduziert, können die Primer an die passenden Stellen in der DNA - sofern diese in der Probe vorhanden sind - binden. Dies erlaubt es der Polymerase, die dahinter liegende DNA zu kopieren.

Diese Schritte werden immer wieder wiederholt, und in jedem Schritt erhöht sich die Menge der DNA in der Probe, falls zu den Primern passende Sequenzen in der DNA in der Probe vorhanden waren. Dieser Vorgang wird als \textbf{Amplifikation} der zwischen den Primer-Sequenzen liegenden Sequenz bezeichnet. Sonst passiert nichts, und es kann keine Erhöhung der DNA-Menge gemessen werden. 
