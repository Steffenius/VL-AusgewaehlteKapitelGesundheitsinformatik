\section{DNA-Sequenzierung}

\subsection{PCR}

Die Fähigkeit von Organismen, ihre DNA zu kopieren, wird in der molekularen Analytik dafür genutzt, DNA zu manipulieren und nachzuweisen. Eine der grundlegenden Methoden dabei ist die \textbf{Polymerase Chain Reaction} (\textbf{PCR}). Die PCR wird verwendet, um gezielt ein bekanntes Stück DNA massiv zu vermehren. Da es möglich ist, die Menge an DNA in einer Probe zu messen, kann der Erfolg dieser Vermehrung gemessen werden. Wird also versucht, in einer Probe gezielt ein Stück DNA zu vermehren, von dem bekannt ist, dass es nur in einem bestimmten Organismus vorhanden ist (z.B. in Ebolaviren), und wird danach gezeigt, dass die Vermehrung erfolgreich war, ist der Beweis erbracht, dass dieser Organismus in der Probe vorhanden ist. 

Die PCR bedient sich der DNA-Polymerase und ihrer zwei definierenden Eigenschaften: Der Fähigkeit, einen DNA-Einzelstrang zu kopieren, und ihrer Abhängigkeit von einem Primer. Um eine PCR durchzuführen, werden zu einer Probe mit unbekannter DNA zunächst DNA-Polymerase sowie Nukleotide\footnote{inklusive einiger weiterer für die Reaktion notwendiger Chemikalien} hinzugefügt. Da keine Primase vorhanden ist, kann die DNA-Polymerase in diesem Zustand noch keine DNA duplizieren. Anstatt von Primase werden nun aber direkt Primer hinzugefügt: Kurze, meist ca. 20 Basen lange DNA-Fragmente, die künstlich synthetisiert wurden und revers-komplementär zu nah beieinander liegenden DNA-Bereichen sind, welche nur in dem nachzuweisenden Organismus vorhanden sind. Wird nun die Temperatur soweit erhöht, dass die DNA-Doppelhelix sich in zwei Einzelstränge löst, und danach wieder ein wenig reduziert, können die Primer an die passenden Stellen in der DNA - sofern diese in der Probe vorhanden sind - binden. Dies erlaubt es der Polymerase, die dahinter liegende DNA zu kopieren.

Diese Schritte werden immer wieder wiederholt, und in jedem Schritt erhöht sich die Menge der DNA in der Probe, falls zu den Primern passende Sequenzen in der DNA in der Probe vorhanden waren. Dieser Vorgang wird als \textbf{Amplifikation} der zwischen den Primer-Sequenzen liegenden Sequenz bezeichnet. Sonst passiert nichts, und es kann keine Erhöhung der DNA-Menge gemessen werden. 


\subsection{Sanger-Sequenzierung}



\subsection{Second Generation Sequencing}

Die klassischen Sequenziermethoden (Sanger-Sequenzierung, Pyrosequenzierung) erlaubten es, die Genome erster Organismen zu sequenzieren und die molekulare Diagnostik von Krankheitserregern masiv zu verbessern. Allerdings ist die Rekonstruktion ganzer Genome, deren Länge zwischen ca. 1800 Basen für die kürzesten Viren über 1-10 Millionen Basen für die meisten Bakterien, ca. 3 Milliarden Basen für den Menschen, bis hin zu mehreren hundert Milliarden Basen bei einigen Organismen\footnote{Das längste bekannte Genom wird auf ca. 670 Milliarden Basen bei einer Amöbe geschätzt.} beträgt, mit der Sequenzierung einzelner maximal 1000 Basen langer Stücke extrem aufwändig. Das erste menschliche Genom wurde mit dieser Methode sequenziert, und die Kosten beliefen sich auf ca. 300 Millionen USD. 

Seit den 1990er Jahren wurde entsprechend an der Parallelisierung der Sequenzierung gearbeitet. Die daraus erwachsenden Sequenziermethoden, die es erlaubten, zunächst einige zig Tausend, später Milliarden von Sequenzierreaktionen gleichzeitig durchzuführen, werden als \textbf{Second Generation Sequencing} bezeichnet. Das Vorgehen beim Second Generation Sequencing orientiert sich dabei grundsätzlich an den Methoden des First Generation Sequencing, modifiziert dieses aber, um die hohe Parallelisierung zu ermöglichen. Diese Parallelisierung senkt die Kosten für die Sequenzierung vollständiger Genome massiv - ein komplettes menschliches Genom kann heutzutage mittels Second Generation Sequencing für unter 1000 USD sequenziert werden. 

Eine Gemeinsamkeit aller Second Generation Sequencing-Methoden ist, dass die DNA zunächst aufbereitet werden muss. Dieser Vorgang wird als \textbf{library preparation} bezeichnet. Dabei wird die komplette in einer Probe vorhandene DNA zufällig in kurze Stücke zerschnitten\footnote{Dabei kommen unterschiedliche Methoden zum Einsatz, z.B. mechanisches Zerreißen durch Beschallung mit Ultraschall.} und an diese Stücke werden an beiden Seiten mit spezialisierten Enzymen vorab synthetisierte zwei kurze DNA-Stücke mit bekannter Sequenz, genannt \textbf{Adapter}, gebunden. Diese sind notwendig, da bei der Sequenzierung selber wieder Polymerase und Primer zum Einsatz kommen. Es lassen sich aber nicht für Millionen von Reaktionen kosteneffizient einzeln spezialisierte Primer erstellen. Durch die Erweiterung der unbekannten Genom-Bruchstücke - als \textbf{Fragmente} bezeichnet - um Adapter mit bekannten Sequenzen kann die Sequenzierung mit nur zwei Primern - passend jeweils zu den zwei verwendeten Adaptern - durchgeführt werden. Ziel ist es nun wieder, die einzelnen Fragmente zu vervielfältigen und dabei Signale zu generieren, die eine Rekonstruktion der Sequenzen ermöglichen. Das Kopieren eines einzelnen Fragments liefert allerdings kein detektierbares Signal. Deshalb ist bei First Generation Sequencing-Methoden ein PCR-Produkt das Ausgangsmaterial, und deshalb muss auch beim Second Generation Sequencing zunächst eine Amplifikation jedes Fragments vorgenommen werden - für Millionen von Fragmenten, ohne, dass die Produkte durcheinandergemischt werden, da dann kein klar zuordenbares Signal mehr generiert werden könnte. Das Vorgehen unterscheidet sich dabei von Technologie zu Technologie. 

\subsubsection{Pyrosequenzierung-basiertes NGS: 454, IonTorrent}

Eine populäre Methode des NGS basiert auf der Pyrosequenzierung. Zu der Mischung mit - wie oben beschrieben erstellten - von Adaptern flankierten DNA-Fragmenten werden mikroskopisch als \textbf{Beads} bezeichnete Kügelchen hinzugegeben. An jedem dieser Kügelchen sind viele kurze, zu einem Adapter revers-komplementäre, DNA-Stücke befestigt. Wie Primer können diese DNA-Stücke an den Adapter-Sequenzen binden. Wird das Verhältnis der Anzahl eingesetzter Beads zur Anzahl an DNA-Fragmenten in der Probe korrekt gewählt, bindet an die meisten Beads jeweils genau ein DNA-Fragment. Die Beads werden daraufhin in eine Mischung von Öl und Wasser mit allen für eine PCR notwendigen Reagenzien übertragen und geschüttelt, so dass eine Emulsion entsteht. Da die Beads wasserlöslich sind, bleiben sie in den Wassertropfen - aufgrund der Größe der Beads in den meisten Fällen jeweils nur ein Bead in einem Wassertropfen. Werden nun die üblichen Zyklen einer PCR durchgeführt, findet die Reaktion in jedem der Wassertropfen unabhängig statt. Da DNA wasserlöslich ist, bleiben die Produkte der einzelnen PCR-Reaktionen - deren Ausgangsmaterial jeweils ein einzelnes Fragment war - jeweils in ihren Wassertropfen und somit räumlich voneinander getrennt. Und da die Adapter mitkopiert werden, bleiben die neuen Kopien auch an den DNA-Fragmenten am jeweiligen Bead hängen. So wird sichergestellt, dass auch später alle zusammengehörigen DNA-Fragmente gemeinsam an einem Bead bleiben. 

Nach diesem Vervielfältigungsvorgang werden die Beads auf eine \textbf{Mikrotiterplatte} gegeben, in der sich je nach Gerät mehrere hunderttausend bis Millionen mikroskopischer Vertiefungen befinden. In jede Vertiefung passt exakt ein Bead mit der an ihn gebundenen DNA. Wird nun auf der gesamten Platte eine Pyrosequenzierung durchgeführt, kann bei der Zugabe jedes Nukleotids ein Bild von der Platte aufgenommen werden. Die Helligkeit der einzelnen Pixel ist dabei repräsentativ für das in jeder Vertiefung generierte Lichtsignal und es kann entsprechend aus der Bildfolge für jede Vertiefung die Sequenz des Fragments darin rekonstruiert werden.  

\subsubsection{Sanger-basiertes NGS: Illumina}

\subsection{Third Generation Sequencing: Nanopore (MinION), SMRT (PacBio)}

\subsection{Empfohlene Youtube-Videos}
\begin{description}[align=left]
	\item [PCR] \href{https://www.youtube.com/watch?v=iQsu3Kz9NYo}{https://www.youtube.com/watch?v=iQsu3Kz9NYo}
	\item [Sanger-Sequenzierung] \href{https://www.youtube.com/watch?v=ONGdehkB8jU}{https://www.youtube.com/watch?v=ONGdehkB8jU}
	\item [454-Sequenzierung] \href{https://www.youtube.com/watch?v=KzdWZ5ryBlA}{https://www.youtube.com/watch?v=KzdWZ5ryBlA}
	\item [Illumina-Sequenzierung] \href{https://www.youtube.com/watch?v=fCd6B5HRaZ8}{https://www.youtube.com/watch?v=fCd6B5HRaZ8}
	\item [Nanopore-Sequenzierung] \href{https://www.youtube.com/watch?v=E9-Rm5AoZGw}{https://www.youtube.com/watch?v=E9-Rm5AoZGw}
	\item [SMRT-Sequenzierung] \href{https://www.youtube.com/watch?v=NHCJ8PtYCFc}{https://www.youtube.com/watch?v=NHCJ8PtYCFc}
\end{description}

\subsection{Kontrollfragen}
\begin{enumerate}
	\item Wofür werden bei der PCR Primer benötigt?
	\item Werden bei der Sanger- oder bei der Pyrosequenzierung fluoreszent markierte Nukleotide benötigt? Welche weitere Eigenschaft (neben der fluoreszenten Markierung) haben diese modifizierten Nukleotide?
\end{enumerate}
