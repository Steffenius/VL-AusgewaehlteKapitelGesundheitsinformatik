\section{DNA-Sequenzierung}

\subsection{PCR}

Die Fähigkeit von Organismen, ihre DNA zu kopieren, wird in der molekularen Analytik dafür genutzt, DNA zu manipulieren und nachzuweisen. Eine der grundlegenden Methoden dabei ist die \textbf{Polymerase Chain Reaction} (\textbf{PCR}). Die PCR wird verwendet, um gezielt ein bekanntes Stück DNA massiv zu vermehren. Da es möglich ist, die Menge an DNA in einer Probe zu messen, kann der Erfolg dieser Vermehrung gemessen werden. Wird also versucht, in einer Probe gezielt ein Stück DNA zu vermehren, von dem bekannt ist, dass es nur in einem bestimmten Organismus vorhanden ist (z.B. in Ebolaviren), und wird danach gezeigt, dass die Vermehrung erfolgreich war, ist der Beweis erbracht, dass dieser Organismus in der Probe vorhanden ist. 

Die PCR bedient sich der DNA-Polymerase und ihrer zwei definierenden Eigenschaften: Der Fähigkeit, einen DNA-Einzelstrang zu kopieren, und ihrer Abhängigkeit von einem Primer. Um eine PCR durchzuführen, werden zu einer Probe mit unbekannter DNA zunächst DNA-Polymerase sowie Nukleotide\footnote{inklusive einiger weiterer für die Reaktion notwendiger Chemikalien} hinzugefügt. Da keine Primase vorhanden ist, kann die DNA-Polymerase in diesem Zustand noch keine DNA duplizieren. Anstatt von Primase werden nun aber direkt Primer hinzugefügt: Kurze, meist ca. 20 Basen lange DNA-Fragmente, die künstlich synthetisiert wurden und revers-komplementär zu nah beieinander liegenden DNA-Bereichen sind, welche nur in dem nachzuweisenden Organismus vorhanden sind. Wird nun die Temperatur soweit erhöht, dass die DNA-Doppelhelix sich in zwei Einzelstränge löst, und danach wieder ein wenig reduziert, können die Primer an die passenden Stellen in der DNA - sofern diese in der Probe vorhanden sind - binden. Dies erlaubt es der Polymerase, die dahinter liegende DNA zu kopieren.

Diese Schritte werden immer wieder wiederholt, und in jedem Schritt erhöht sich die Menge der DNA in der Probe, falls zu den Primern passende Sequenzen in der DNA in der Probe vorhanden waren. Dieser Vorgang wird als \textbf{Amplifikation} der zwischen den Primer-Sequenzen liegenden Sequenz bezeichnet. Sonst passiert nichts, und es kann keine Erhöhung der DNA-Menge gemessen werden. 


\subsection{Sanger-Sequenzierung}



\subsection{Second Generation Sequencing}

Die klassischen Sequenziermethoden (Sanger-Sequenzierung, Pyrosequenzierung) erlaubten es, die Genome erster Organismen zu sequenzieren und die molekulare Diagnostik von Krankheitserregern masiv zu verbessern. Allerdings ist die Rekonstruktion ganzer Genome, deren Länge zwischen ca. 1800 Basen für die kürzesten Viren über 1-10 Millionen Basen für die meisten Bakterien, ca. 3 Milliarden Basen für den Menschen, bis hin zu mehreren hundert Milliarden Basen bei einigen Organismen\footnote{Das längste bekannte Genom wird auf ca. 670 Milliarden Basen bei einer Amöbe geschätzt.} beträgt, mit der Sequenzierung einzelner maximal 1000 Basen langer Stücke extrem aufwändig. Das erste menschliche Genom wurde mit dieser Methode sequenziert, und die Kosten beliefen sich auf ca. 300 Millionen USD. 

Seit den 1990er Jahren wurde entsprechend an der Parallelisierung der Sequenzierung gearbeitet. Die daraus erwachsenden Sequenziermethoden, die es erlaubten, zunächst einige zig Tausend, später Milliarden von Sequenzierreaktionen gleichzeitig durchzuführen, werden als \textbf{Second Generation Sequencing} bezeichnet. Das Vorgehen beim Second Generation Sequencing orientiert sich dabei grundsätzlich an den Methoden des First Generation Sequencing, modifiziert dieses aber, um die hohe Parallelisierung zu ermöglichen. Diese Parallelisierung senkt die Kosten für die Sequenzierung vollständiger Genome massiv - ein komplettes menschliches Genom kann heutzutage mittels Second Generation Sequencing für unter 1000 USD sequenziert werden. 

Eine Gemeinsamkeit aller Second Generation Sequencing-Methoden ist, dass die DNA zunächst aufbereitet werden muss. Dieser Vorgang wird als \textbf{library preparation} bezeichnet. Dabei wird die komplette in einer Probe vorhandene DNA zufällig in kurze Stücke zerschnitten\footnote{Dabei kommen unterschiedliche Methoden zum Einsatz, z.B. mechanisches Zerreißen durch Beschallung mit Ultraschall.} und an diese Stücke werden an beiden Seiten mit spezialisierten Enzymen vorab synthetisierte zwei kurze DNA-Stücke mit bekannter Sequenz, genannt \textbf{Adapter}, gebunden. Diese sind notwendig, da bei der Sequenzierung selber wieder Polymerase und Primer zum Einsatz kommen. Es lassen sich aber nicht für Millionen von Reaktionen kosteneffizient einzeln spezialisierte Primer erstellen. Durch die Erweiterung der unbekannten Genom-Bruchstücke - als \textbf{Fragmente} bezeichnet - um Adapter mit bekannten Sequenzen kann die Sequenzierung mit nur zwei Primern - passend jeweils zu den zwei verwendeten Adaptern - durchgeführt werden. Ziel ist es nun wieder, die einzelnen Fragmente zu vervielfältigen und dabei Signale zu generieren, die eine Rekonstruktion der Sequenzen ermöglichen. Eine aus einem einzelnen Fragment generierte Sequenz wird dabei als \textbf{Read} bezeichnet. Das Kopieren eines einzelnen Fragments liefert allerdings kein detektierbares Signal. Deshalb ist bei First Generation Sequencing-Methoden ein PCR-Produkt das Ausgangsmaterial, und deshalb muss auch beim Second Generation Sequencing zunächst eine Amplifikation jedes Fragments vorgenommen werden - für Millionen von Fragmenten, ohne, dass die Produkte durcheinandergemischt werden, da dann kein klar zuordenbares Signal mehr generiert werden könnte. Das Vorgehen unterscheidet sich dabei von Technologie zu Technologie. 

\subsubsection{Pyrosequenzierung-basiertes NGS: 454, IonTorrent}

Eine populäre Methode des NGS basiert auf der Pyrosequenzierung. Zu der Mischung mit - wie oben beschrieben erstellten - von Adaptern flankierten DNA-Fragmenten werden mikroskopisch als \textbf{Beads} bezeichnete Kügelchen hinzugegeben. An jedem dieser Kügelchen sind viele kurze, zu einem Adapter revers-komplementäre, DNA-Stücke befestigt. Wie Primer können diese DNA-Stücke an den Adapter-Sequenzen binden. Wird das Verhältnis der Anzahl eingesetzter Beads zur Anzahl an DNA-Fragmenten in der Probe korrekt gewählt, bindet an die meisten Beads jeweils genau ein DNA-Fragment. Die Beads werden daraufhin in eine Mischung von Öl und Wasser mit allen für eine PCR notwendigen Reagenzien übertragen und geschüttelt, so dass eine Emulsion entsteht. Da die Beads wasserlöslich sind, bleiben sie in den Wassertropfen - aufgrund der Größe der Beads in den meisten Fällen jeweils nur ein Bead in einem Wassertropfen. Werden nun die üblichen Zyklen einer PCR durchgeführt, findet die Reaktion in jedem der Wassertropfen unabhängig statt. Da DNA wasserlöslich ist, bleiben die Produkte der einzelnen PCR-Reaktionen - deren Ausgangsmaterial jeweils ein einzelnes Fragment war - jeweils in ihren Wassertropfen und somit räumlich voneinander getrennt. Und da die Adapter mitkopiert werden, bleiben die neuen Kopien auch an den DNA-Fragmenten am jeweiligen Bead hängen. So wird sichergestellt, dass auch später alle zusammengehörigen DNA-Fragmente gemeinsam an einem Bead bleiben. 

Nach diesem Vervielfältigungsvorgang werden die Beads auf eine \textbf{Mikrotiterplatte} gegeben, in der sich je nach Gerät mehrere hunderttausend bis Millionen mikroskopischer Vertiefungen befinden. In jede Vertiefung passt exakt ein Bead mit der an ihn gebundenen DNA. Wird nun auf der gesamten Platte eine Pyrosequenzierung durchgeführt, kann bei der Zugabe jedes Nukleotids ein Bild von der Platte aufgenommen werden. Die Helligkeit der einzelnen Pixel ist dabei repräsentativ für das in jeder Vertiefung generierte Lichtsignal und es kann entsprechend aus der Bildfolge für jede Vertiefung die Sequenz des Fragments darin rekonstruiert werden.  

Diese Sequenziermethode wurde ursprünglich von der (zwischenzeitlich von Roche aufgekauften und mittlerweile aufgelösten) Firma 454 Life Sciences auf den Markt gebracht und wird in leicht modifizierter Form aktuell in von der Firma Ion Torrent Systems Inc. (mittlerweile von Life Technologies aufgekauft) hergestellten Personal Genome Machine (PGM) genannten Sequenziergeräten eingesetzt. Es ist damit derzeit möglich, in einem weniger als einen Tag dauernden Sequenzierlauf ca. 4 Millionen Reads je ca. 400 Basen Länge (also ca. $1.6*10^9$ Basen an Sequenzinformation) zu generieren. 


\subsubsection{Sanger-basiertes NGS: Illumina}

Eine andere, derzeit marktbeherrschende Sequenziermethode ist die Illumina-Sequenzierung. Anstatt durch das Binden an Beads findet hier die räumlich begrenzte Vervielfältigung der Fragmente mittels \textbf{bridge amplification} statt. Die Fragmente werden über eine Glasplatte gespült, auf der - ähnlich wie bei Beads - zu beiden Adaptern revers-komplementäre DNA-Stücke gebunden sind. Zufällig bleiben die Fragmente als an unterschiedlichen Stellen auf der Platte hängen. Nun wird eine klassische PCR durchgeführt. Die DNA-Stücke auf der Glasplatte fuktionieren dabei als Primer. Nachdem ein Fragment komplett kopiert wurde, kann der - ebenfalls mitkopierte - Adapter auf dem anderen Ende des Fragments wieder an ein anderes revers-komplementäres Stück DNA auf der Glasplatte binden und in der nächsten Runde der PCR erneut kopiert werden. Dieses Brückenschlagen der Fragmente, die auf der einen Seite aus dem in der vorhergehenden PCR-Runde als Primer funktionierenden auf der Glasplatte fixierten DNA-Fragment hervorgehen und auf der anderen Seite an das zum anderen Adapter revers-komplemenräte DNA-Fragment auf der Glasplatte binden (welches in der nächsten PCR-Runde als Primer funktionieren wird) gibt dem Vorgang seinen Namen.

Nach dieser Amplifikation ist aus jedem Fragment, das irgendwo an der Platte gebunden hat, ein ganzes Büschel an identischen DNA-Fragmenten geworden. Diese werden nun sequenziert, indem - ähnlich wie bei der Sanger-Sequenzierung - zu einem der Adapter passende Primer, Polymerase und Nukleotide hinzugefügt werden. Allerdings sind, im Gegensatz zur Sanger-Sequenzierung, alle Nukleotide fluoreszent markiert und so Modifiziert, dass sie nicht mehr von der Polymerase erweitert werden können. Entsprechend kann die Polymerase an jedem der Fragmente nur exakt ein Nukleotid hinter dem Primer einfügen. Danach werden die überschüssigen Nukleotide weggewaschen und es wird ein Bild der Platte aufgenommen. An jeder Stelle, an der ein DNA-Fragment-Büschel vorhanden ist, ist auf dem Bild ein farbiger Punkt zu sehen\footnote{Streng genommen werden vier schwarz-weiß-Bilder aufgenommen, jeweils mit einem Farbfilter, der nur die Farbe einer der fluoreszenten Markierungen durchlässt.}. Aus der Farbe kann geschlossen werden, welche Base an diesem Ort eingebaut wurde. Nun kann durch Zugabe entsprechender Chemikalien die Modifikation der Nukleotide aufgehoben werden, so dass die Polymerase hinter dem neu eingebauten Nukleotid weitere Nukleotide anhängen kann und eine weitere Runde mit neuen modifizierten Nukleotiden (von denen wieder nur jeweils eins eingebaut werden kann) gestartet werden kann. Wird dieser Vorgang immer wieder wiederholt, entsteht bei jedem Durchgang ein neues Bild. Aus der Abfolge von Farben an der gleichen Position in aufeinanderfolgenden Bildern kann wieder die Sequenz jedes Fragments rekonstruiert werden.

Mittels Illumina-Sequenzierung ist es derzeit möglich, in einem wenige Tage dauernden Sequenzierlauf ca. 2 Milliarden Reads je 250 Basen länge (also ca. $5*10^{11}$ Basen an Sequenzinformation) zu generieren.

\subsection{Third Generation Sequencing: Nanopore (MinION), SMRT (PacBio)}

Die Entwicklung der Second Generation Sequencing-Methoden erlaubte einen rasanten Anstieg der Anzahl der bekannten Genome. GenBank, die größte und wichtigste Datenbank von Nukleotidsequenzen, startete 1982 mit 606 Einträgen ($6.8 * 10^5$ Basen). 1990 waren es noch 41 Tausend mittels Sanger-Sequenzierung generierte Einträge ($5.1*10^7$ Basen), 2000 - 18 Jahre nach ihrem Start - dank der Verwendung der ersten Second Generation Sequencing-Geräte schon 5.7 Millionen Einträge ($5.8 * 10^9$ Basen). 2019, weitere 19 Jahre später, sind es 367 Millionen Einträge ($3.2 * 10^{11}$ Basen), die fast alle mittels Second Generation Sequencing erstellt wurden. 

Trotzdem bringt das Second Generation Sequencing einige Nachteile mit sich. Die notwendige Amplifikation der Fragmente während der Library Generation ist zeitaufwändig und kann zu Fehlern führen: Genauso wie die Polymerase beim Kopieren der DNA während der Zellteilung Fehler machen kann, die zu Mutationen führen, kann sie auch bei der PCR Fehler machen, die dann zu falschen Sequenzierungsergebnissen führen\footnote{Der PCR-Schritt birgt noch viele weitere Fehlerquellen, die die spätere Auswerung massiv erschweren, deren Behandlung aber den Rahmen dieser Veranstaltung sprengen würde.}. Außerdem sind die kurzen Readlängen von weniger als 1000 Basen/Read bei der Rekonstruktion der Genome problematisch, wie in dem nächsten Kapitel deutlich wird.

Dies motivierte die Entwicklung einer neuen Generation von Sequenziermethoden, \textbf{Third Generation Sequencing} genannt (gemeinsam mit dem Second Generation Sequencing werden diese Methoden auch als \textbf{Next Generation Sequencing} bezeichnet). Diese haben gemeinsam, dass sie kein Zerschneiden und keine Amplifikation der DNA vor der Sequenzierung erfordern. Stattdessen werden komplette einzelne DNA-Moleküle gelesen. Allerdings liegt die Fehlerrate bei diesen Methoden auch deutlich höher, als bei den Second Generation Sequencing-Methoden: Während dort schon von einer schlechten Qualität geredet wird, wenn $1\%$ der Basen falsch ist und hochqualitative Sequenzierergebnisse nur Fehlerraten von unter $0.01\%$ haben, können bei Third Generation Sequencing aktuell ca. $10\%$ der Basen nicht richtig gelesen werden. Zudem können deutlich weniger Reads generiert werden (weniger als 100 Tausend Reads pro Sequenzierung im Vergleich zu mehreren Millionen bis Milliarden Reads bei Second Generation Sequencing) und die Kosten pro sequenzierte Base sind deutlich höher. Dafür können aber mit Third Generation Sequencing Reads mit Längen von mehreren zig Tausend Basen generiert werden, was die mit Second Generation Sequencing möglichen Read-Längen je nach Technologie um 1-3 Größenordnungen übersteigt. 

Die zwei derzeit auf dem Markt verfügbaren Third Generation Sequencing-Technologien sind das \textbf{Nanopore Sequencing} und das \textbf{Single Molecule Real-Time Sequencing} (\textbf{SMRT}).

\subsubsection{Nanopore Sequencing}

Beim Nanopore Sequencing wird eine Kammer verwendet, die von einer von Poren durchsetzten Membran in zwei Teile getrennt ist. Auf der einen Seite der Membran befindet sich eine Lösung mit einer hohen Konzentration an gleich geladenen Ionen, auf der anderen Seite die gleiche Lösung ohne Ionen. Dieses Ungleichgewicht strebt von Natur aus danach, ausgeglichen zu werden, indem Ionen dem Ladungsgefälle folgend durch die Poren strömen. Der dadurch entstehende Stromfluss kann direkt an den Poren gemessen werden. Wird nun DNA in die Lösung auf einer Seite der Membran gegeben, kann sie von Proteinen, die auf den Poren immobilisiert sind, gebunden und durch die Pore geschoben werden. Da die einzelnen Nukleotide aufgrund ihrer unterschiedlichen chemischen Struktur auch unterschiedlich groß sind, wird die Pore dabei abhängig davon, welche Nukleotide gerade durch sie geschoben werden, unterschiedlich stark verstopft. Je nach Verstopfungsgrad können entsprechend auch unterschiedlich viele Ionen daneben durch die Pore fließen, die Intensität des durch die Pore fließenden Stroms wird also verändert. Aus dieser Veränderung der Stromintensität über die Zeit hinweg wird rekonstruiert, in welcher Reihenfolge Nukleotide durch die Pore geschoben wurden und damit auch die Sequenz des DNA-Moleküls.

\subsubsection{Single Molecule Real-Time Sequencing}

Beim SMRT Sequencing wird im Gegensatz zum Nanopore Sequencing weiterhin eine Polymerase ähnlich wie beim Illumina-Sequencing verwendet. Allerdings ist hier die Signalgenerierung soweit optimiert, dass das Kopieren eines einzelnen Moleküls beobachtet werden kann. In mikroskopisch kleinen Vertiefungen in einer Glasplatte ist jeweils auf dem Boden eine Polymerase immobilisiert. Wird DNA zusammen mit fluoreszent markierten (aber im Gegensatz zur Illumina-Sequenzierung nicht terminierten, also beliebig hintereinander einbaubaren) Nukleotiden und Primern hinzugegeben\footnote{Der eigentliche Vorgang ist ein wenig komplexer, aber konzeptionell ist es das was geschieht.}, kann die Polymerase mit dem Kopieren beginnen. Dabei wird stetig von unten das fluoreszente Signal von der Stelle, an der sich die Polymerase befindet, aufgenommen. Dabei generieren Nukleotide, die eingebaut werden und entsprechend über die gesamte Zeit des Einbauvorgangs von der Polymerase an einer Stelle gehalten werden, ein lang anhaltendes Lichtsignal. Die Abfolge dieser Lichtsignale wird verwendet, um die Reihenfolge, in der die Polymerase die Nukleotide eingebaut hat, zu rekonstruieren. 

\subsection{Empfohlene Youtube-Videos}
\begin{description}[align=left]
	\item [PCR] \href{https://www.youtube.com/watch?v=iQsu3Kz9NYo}{https://www.youtube.com/watch?v=iQsu3Kz9NYo}
	\item [Sanger-Sequenzierung] \href{https://www.youtube.com/watch?v=ONGdehkB8jU}{https://www.youtube.com/watch?v=ONGdehkB8jU}
	\item [454-Sequenzierung] \href{https://www.youtube.com/watch?v=KzdWZ5ryBlA}{https://www.youtube.com/watch?v=KzdWZ5ryBlA}
	\item [Illumina-Sequenzierung] \href{https://www.youtube.com/watch?v=fCd6B5HRaZ8}{https://www.youtube.com/watch?v=fCd6B5HRaZ8}
	\item [Nanopore-Sequenzierung] \href{https://www.youtube.com/watch?v=E9-Rm5AoZGw}{https://www.youtube.com/watch?v=E9-Rm5AoZGw}
	\item [SMRT-Sequenzierung] \href{https://www.youtube.com/watch?v=NHCJ8PtYCFc}{https://www.youtube.com/watch?v=NHCJ8PtYCFc}
\end{description}

\subsection{Kontrollfragen}
\begin{enumerate}
	\item Wofür werden bei der PCR Primer benötigt?
	\item Werden bei der Sanger- oder bei der Pyrosequenzierung fluoreszent markierte Nukleotide benötigt? Welche weitere Eigenschaft (neben der fluoreszenten Markierung) haben diese modifizierten Nukleotide?
	\item Weshalb ist eine Amplifikation der einzelnen Fragmente bei Second Generation Sequencing notwendig?
	\item Weshalb werden bei der Library Preparation die DNA-Stücke mit unbekannter Sequenz mit Adaptern mit bekannter Sequenz verbunden?
	\item Was sind Vor- und Nachteile von Second Generation Sequenziertechnologien im Vergleich zu Third Generation Sequencing-Technologien?
\end{enumerate}
